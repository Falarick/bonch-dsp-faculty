\documentclass{beamer}

\mode<presentation> 
{
    \usetheme{Madrid}
}

\usepackage[utf8x]{inputenc}
\usepackage[english,russian]{babel}
\usepackage[T2A]{fontenc}
\usepackage{graphicx}
\usepackage{booktabs} 
\usepackage{mathtools}
\usepackage{amsmath}
\usepackage{wasysym}
\usepackage{subfig}
\usepackage{hyperref}
\usepackage{ulem}
\usepackage{ragged2e}
\usepackage{algorithm2e}
\usepackage{minted}


\DeclarePairedDelimiter\ceil{\lceil}{\rceil}
\DeclarePairedDelimiter\floor{\lfloor}{\rfloor}


\usemintedstyle{borland}

\usefonttheme[onlymath]{serif}

\hypersetup
{
    colorlinks=true,
    linkcolor=white, 
    urlcolor=cyan
}

\title[Лекция 7]
{
    Лекция 7: Указатели на функцию в языке Си и создание проектов
} 


\author[Д. А. Караваев]{Д. А. Караваев}

\institute[СПбГУТ] 
{
    Санкт-Петербургский государственный университет телекоммуникаций \\ им. проф. М. А. Бонч-Бруевича \\ 
    \vspace{0.2cm}
    Факультет РТС, Кафедра РОС \\
    \vspace{0.2cm}
    Факультатив <<Программирование в ЦОС>> \\
    \vspace{0.2cm}
    Осень 2019
}

\date[16.12.2019]{16.12.2019 Санкт-Петербург} 

\begin{document}
    \begin{frame}
        \titlepage 
    \end{frame}
    \begin{frame}[fragile]
        \frametitle{Функция как параметр}
        \justifying
        Многие алгоритмы можно сделать зависящими от {\it функционального} параметра во избежание {\bf дублирования} кода:
        \begin{itemize}
            \item Алгоритм сортировки: функция сравнения ({\it предикат}) $\implies$ сортировка в возрастающем/убывающем порядке;
            \item Операция {\tt Reduce}: "обобщённое"\ суммирование (c использование бинарной функции);
            \item Операция {\tt Map}: преобразование массива по определенном правилу (унарной функции);
            \item Обобщённый цикл {\tt ForEach};
            \item $\dotsc$
        \end{itemize}
        \par
        {\bf Вопрос}: Как передать функцию в качестве аргумента другой функции?
    \end{frame}
    \begin{frame}[fragile]
        \frametitle{Указатели на функцию}
        \begin{minted}[frame=lines,         framesep=2mm,
                       baselinestretch=1.2, fontsize=\footnotesize,
                       linenos]{c}
/* Синтаксис: <возвращаемый тип> (* <имя>)(<типы аргументов>): */
/* Некоторая функция: */
float sum(float a, float b)
{
    return a + b;
}
/* Объявление указателя на функцию: */
float (*sum_another)(float, float) = sum;
/* Указатель на функцию sum_over - псевдоним функции sum! */
/* Пример использования: */
float x = sum_another(1.0, 2.0);
printf("(sum == sum_another) == %d", (x == sum(1.0, 2.0)));
/* Результат: (sum == sum_another) == 1. */
        \end{minted}
    \end{frame}
    \begin{frame}[fragile]
        \frametitle{Передача указателя на функцию}
        \begin{minted}[frame=lines,         framesep=2mm,
                       baselinestretch=1.2, fontsize=\footnotesize,
                       linenos]{c}
/* Функция в качестве аргумента: */
void merge_sort(int *a, size_t N, int *(comparator)(int, int));
/* Объявляем "сравниватели": */
int less(int lhs, int rhs)
{
    return a < b;
}
int more(int lhs, int rhs)
{
    return a > b;
}
/* Сортировка по убыванию: */
merge_sort(a, N, less);
/* Сортировка по возрастанию: */
merge_sort(a, N, more);
        \end{minted}
    \end{frame}
    \begin{frame}[fragile]
        \frametitle{Функциональный тип}
        \begin{minted}[frame=lines,         framesep=2mm,
                       baselinestretch=1.2, fontsize=\footnotesize,
                       linenos]{c}
/* Тип функции с заданными аргументами и значением (!интерфейс!): */
typedef float *(binary_op)(float, float);
/* Функции удовлетворяющие типу binary_op: */
float sum(float a, float b)  { /* ... */ }
float prod(float a, float b) { /* ... */ }
/* "Свернуть" массив по правилу: */
float reduce(float *a, size_t N, binary_op bop, float base)
{
    float ret = base;
    for (size_t i = 0; i < N; ++i)
    {
        ret = bop(ret, a[i]);
    }
    return ret;
}
        \end{minted}
    \end{frame}
    \begin{frame}[fragile]
        \frametitle{Вычисление определённого интеграла}
        \justifying
        Необходимо вычислить определённый интеграл на промежутке $[a, b]$ от произвольной функции $f(x)$:
        \begin{equation}
            S = \int\limits_{a}^{b} f(x) \textrm{d}x
        \end{equation}
        \justifying
        Для этого пользуются геометрической интерпретацией интеграла как площади кривой под функцией $f(x)$. 
        \par
        \vspace{0.2cm}
        \justifying
        Существует большое множество различный методов {\it численного} интегрирования (Самарский А. А. Введение в численные методы). 
    \end{frame}
    \begin{frame}[fragile]
        \frametitle{Метод прямоугольников}
        \justifying
        Интервал $[a, b]$ разбивается на {\it равномерную сетку} с шагом $d$: 
        $$d = (b - a) / N,\ \{x_{i} = a + id\},\ i \in \{0, \dotsc, N - 1\},$$
        $$S \approx \int\limits_{a}^{b} f(x) \textrm{d}x = d\sum_{i=0}^{N-1}f(x_{i}).$$
        \par
        \justifying
        Т.е. функция $f(x)$ {\it интерполируется} на интервале $(x_{i}, x_{i+1})$ константами (горизонтальными прямыми) (0-ой порядок аппроксимации).
    \end{frame}
    \begin{frame}[fragile]
        \frametitle{Метод трапеций}
        \justifying
        Интервал $[a, b]$ разбивается на {\it равномерную сетку} с шагом $d$: 
        $$d = (b - a) / N,\ \{x_{i} = a + id\},\ i \in \{0, \dotsc, N - 1\},$$
        $$S \approx \int\limits_{a}^{b} f(x) \textrm{d}x = d\sum_{i=0}^{N-1}\frac{f(x_{i}) + f(x_{i + 1})}{2}.$$
        \par
        \justifying
        Функция $f(x)$ {\it интерполируется} на интервале $(x_{i}, x_{i+1})$ прямыми (1-ый порядок аппроксимации).
    \end{frame}
    \begin{frame}[fragile]
        \frametitle{Метод Симпсона (парабол)}
        \justifying
        Интервал $[a, b]$ разбивается на {\it равномерную сетку} с шагом $d$: 
        $$d = (b - a) / N,\ \{x_{i} = a + id\},\ i \in \{0, \dotsc, N - 1\},$$
        $$S \approx \int\limits_{a}^{b} f(x) \textrm{d}x = \frac{d}{6}\sum_{i=0}^{N-1}\big(f(x_{i}) + 4f\Big(\frac{f(x_{i}) + f(x_{i+1})}{2}\Big) + f(x_{i + 1})\big).$$
        \par
        \justifying
        Функция $f(x)$ {\it интерполируется} на интервале $(x_{i}, x_{i+1})$ параболами (2-ой порядок аппроксимации).
    \end{frame}
    \begin{frame}[fragile]
        \frametitle{Метод Монте-Карло}
        \justifying
        Пусть $u$ - случайная велечина, распределённая равномерно на $[a, b] \implies f(u)$ так же случайная величина со средним:
        $$\textrm{E}\big\{f(u)\big\} = \int\limits_{a}^{b}f(x)\varphi(x)\textrm{d}x = \frac{1}{b-a}\int\limits_{a}^{b}f(x)\textrm{d}x,$$
        $$\int\limits_{a}^{b}f(x)\textrm{d}x = (b - a)\textrm{E}\big\{f(u)\big\} \approx \frac{b - a}{N}\sum_{i=0}^{N-1}f(\hat{u}_{i}),$$
        где $\hat{u}_{i},\ i \in \{0, \dotsc, N - 1\}$ - конкретные значения случайной величины $u$ (выборка размера $N$).
    \end{frame}
    \begin{frame}[fragile]
        \frametitle{Задание}
        \justifying
        Реализовать один из 4-х методов численного интегрирования в виде функции на языке Си, которая принимает $f(x)$ в качестве параметра:
        \begin{minted}[frame=lines,         framesep=2mm,
                       baselinestretch=1.2, fontsize=\footnotesize,
                       linenos]{c}
/* Тип функции: */
typedef float *(unary_function)(float, float);
/* Прототип функции: */
float num_integral(float a, float b, unary_function f, size_t N);
        \end{minted}
        \par
        \vspace{0.2cm}
        \justifying
        {\bf NB}: Для того чтобы реализовать вычисление методом Монте-Карло необходимо воспользоваться стандартной фукнцией {\tt rand}, подключив заголовочный файл {\tt stdlib}: \url{http://www.cplusplus.com/reference/cstdlib/rand/}.
    \end{frame}
    \begin{frame}[fragile]
        \frametitle{Создание проекта}
        \justifying
        Создадим проект для решения задачи численного интегрирования:
        \begin{minted}[frame=lines,        framesep=2mm,
               baselinestretch=1.2, fontsize=\footnotesize,
               linenos]{bash}
# Создадим папку проекта:
mkdir Integral && cd Integration
# Создадим папки для файлов с кодом:
mkdir include source
# Создадим файлы:
touch source/main.c source/integral.c include/integral.h CMakelists.txt
        \end{minted}
        \par
        \justifying
        В языке Си файл с иходным кодом деляться на два типа:
        \begin{enumerate}
            \item Заголовочные ({\tt *.h}) - для объявления типов и прототипов (интерфейсный файлы);
            \item Исходные ({\tt *.c}) - для реализации функций (файлы реализации).  
        \end{enumerate} 
    \end{frame}
    \begin{frame}[fragile]
        \frametitle{Пример: заголовочный/исходный файл}
        Файл {\tt include/integral.h}:
        \begin{minted}[frame=lines,        framesep=2mm,
               baselinestretch=1.2, fontsize=\footnotesize,
               linenos]{c}
#include <stdlib.h> /* Подключение стадартного заголовочного файла. */
/* Тип функции: */
typedef float *(unary_function)(float, float);
/* Прототип функции: */
float num_integral(float a, float b, unary_function f, size_t N);
        \end{minted}
        Файл {\tt source/integral.c}:
        \begin{minted}[frame=lines,        framesep=2mm,
               baselinestretch=1.2, fontsize=\footnotesize,
               linenos]{c}
#include "integral.h" /* Подключение заголовочного файла. */
/* Определение функции: */
float num_integral(float a, float b, unary_function f, size_t N)
{
    /* ... */
}
        \end{minted}
    \end{frame}
    \begin{frame}[fragile]
        \frametitle{Пример: файл с функцией {\tt main}}
        Теперь функцией {\tt num\_integral} можно воспользоваться в функции {\tt main}:
        \begin{minted}[frame=lines,        framesep=2mm,
               baselinestretch=1.2, fontsize=\footnotesize,
               linenos]{c}
#include <stdio.h> /* Подключение стадартного заголовочного файла. */
#include <math.h> /* Заголовочный файл с математическим функциями. */

#include "integral.h" /* Подключение заголовочного файла. */

int main()
{
    float S = num_integral(0, 3.14 * 5, sin, 1000);
    printf("S{sin(x)} = %f\n", S);
    return 0;
}
        \end{minted}
        \par
        По подобному принципу формируются {\it библиотеки} с исходным кодом, у которых есть некоторое предназначение (библиотека для работы с изображениями {\tt OpenCV}),
    \end{frame}
    \begin{frame}[fragile]
        \frametitle{Пример: файл сборки {\tt CMakeLists.txt}}
        \justifying
        Исходный код в исполняемый файл преобразуют компилятор + линковщик. Прямая работа с этими программами неудобна при большом числе файлов в проекте, поэтому существует системы сборки, такие как {\tt cmakе}: \url{https://cmake.org/}.
        \begin{minted}[frame=lines,        framesep=2mm,
               baselinestretch=1.2, fontsize=\footnotesize,
               linenos]{cmake}
# Создание проекта для сборки с именем:
PROJECT(Integral C) 
# Переменная с именем исполняемого файла:
SET(EXEC_TARGET integral)
# Добавить папку с заголовочными файлами:
INCLUDE_DIRECTORIES(${CMAKE_CURRENT_SOURCE_DIR}/include)
# Переменные со списком заголовочных и исходных файлов:
SET(HEADERS ./include/integral.h)
SET(SOURCES ./source/integral.c ./source/main.c)
# Команда сборки исполняемого файла:
ADD_EXECUTABLE(${EXEC_TARGET} ${SOURCES})
        \end{minted}
    \end{frame}
    \begin{frame}
        \begin{center}
        \baselineskip 20.0mm
        \Huge Спасибо за внимание!
        \end{center}
    \end{frame}
\end{document}
