\documentclass{beamer}

\mode<presentation> 
{
    \usetheme{Madrid}
}

\usepackage[utf8x]{inputenc}
\usepackage[english,russian]{babel}
\usepackage[T2A]{fontenc}
\usepackage{graphicx}
\usepackage{booktabs} 
\usepackage{mathtools}
\usepackage{amsmath}
\usepackage{wasysym}
\usepackage{subfig}
\usepackage{hyperref}
\usepackage{ulem}
\usepackage{ragged2e}
\usepackage{algorithm2e}
\usepackage{minted}

\usemintedstyle{borland}

\usefonttheme[onlymath]{serif}

\hypersetup
{
    colorlinks=true,
    linkcolor=white, 
    urlcolor=cyan
}

\title[Лекция 4]
{
    Лекция 4: Задача сортировка и функции в языке Си (продолжение)
} 


\author[Д. А. Караваев]{Д. А. Караваев}

\institute[СПбГУТ] 
{
    Санкт-Петербургский государственный университет телекоммуникаций \\ им. проф. М. А. Бонч-Бруевича \\ 
    \vspace{0.2cm}
    Факультет РТС, Кафедра РОС \\
    \vspace{0.2cm}
    Факультатив <<Программирование в ЦОС>> \\
    \vspace{0.2cm}
    Осень 2019
}

\date[11.11.2019]{11.11.2019 Санкт-Петербург} 

\begin{document}
    \begin{frame}
        \titlepage 
    \end{frame}
    \begin{frame}
        \frametitle{Аргументы функций}
        \justifying
        Существует два способа передачи аргументов в функцию:
        \begin{enumerate}
            \item {\bf По значению}: в качестве аргументов функции выступают соответствующие {\it копии} значений выражений, переданных в функцию в момент её вызова. Подобный механизм неявно реализован в Си (поведение по-умполчанию).
            \item {\bf По ссылке}: в качестве аргументов функции выступают {\it синонимы} переменных, переданных функцию. Подобный механизм используется в Java. В Си его можно реализовать явно при помощи {\it указателей}!
        \end{enumerate}
        \par
        {\bf Задача}: Что будет если мы захотим изменить значение той переменной, которая была передана в функцию в качестве параметра?
    \end{frame}
    \begin{frame}[fragile]
        \frametitle{Классический пример: функция {\tt swap}}
        \begin{minted}[frame=lines,         framesep=2mm,
                       baselinestretch=1.2, fontsize=\footnotesize,
                       linenos]{c}
/* Реализуем функцию, которая меняет значения двух переменных. */
/* Не правильно (оперируем с копиями, а не с самими перемнными): */
void swap_error(int x, int y)
{
    int temp = x;
    x = y;
    y = temp;
}
/* Правильно: используем указатели (передаём адреса, а не значения): */
void swap_correct(int* x, int* y)
{
    int temp = *x;
    *x = *y; /* (*x) - синоним переменной x! */
    *y = temp;
}
        \end{minted}
    \end{frame}
    \begin{frame}[fragile]
        \frametitle{Задача сортировки}
        \begin{block}{Формулировка}
            \justifying
            {\bf Вход}: Массив $a$ длины $N$;
            \par
            {\bf Выход}: Массив $a$ длины $N$: $a[0] \leq a[1] \leq \dotsc \leq a[N - 1]$ (по убыванию).
        \end{block}
        \par
        \begin{algorithm}[H]
            \DontPrintSemicolon
            \SetKwFunction{FSort}{InsertionSort}
            \SetKwFunction{FMinIndex}{MinIndex}
            \SetKwFunction{FSwap}{Swap}

            \SetKwProg{Fn}{Fuction}{:}{}
            \Fn{\FSort{$a$ : массив, $N$ : длина $a$}}
            {
                \For{$n \leftarrow 0$ \KwTo $N - 1$} 
                {
                    $j \leftarrow$ \FMinIndex($a,\ n,\ N$)\;
                    \FSwap($a[n],\ a[j]$)\;
                }
            }
            \;
            \Fn{\FMinIndex{$a$ : массив, $begin$ : начало, $end$ : конец}}
            {
                // Ваш (псевдо)код тут
            }
            \;
        \end{algorithm}
        \justifying
        \par
        {\bf Замечание}: Простейшим примером сортировки, является сортировка вставками (сложность $O(N^{2})$), основанная на поиске минимума.
    \end{frame}
    \begin{frame}
        \begin{center}
        \baselineskip 20.0mm
        \Huge Спасибо за внимание!
        \end{center}
    \end{frame}
\end{document}
